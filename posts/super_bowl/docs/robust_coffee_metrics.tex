\documentclass{article}

% Packages
\usepackage[utf8]{inputenc}
\usepackage{geometry}
\usepackage{amsmath}
\usepackage{graphicx}
\usepackage{booktabs}
\usepackage{hyperref}
\usepackage{float}
\usepackage{subcaption}
\usepackage{listings}
\usepackage{xcolor}

% Code Listing Configuration
\definecolor{codegreen}{rgb}{0,0.6,0}
\definecolor{codegray}{rgb}{0.5,0.5,0.5}
\definecolor{codepurple}{rgb}{0.58,0,0.82}
\definecolor{backcolour}{rgb}{0.95,0.95,0.92}

\lstdefinestyle{mystyle}{
    backgroundcolor=\color{backcolour},   
    commentstyle=\color{codegreen},
    keywordstyle=\color{magenta},
    numberstyle=\tiny\color{codegray},
    stringstyle=\color{codepurple},
    basicstyle=\ttfamily\footnotesize,
    breakatwhitespace=false,         
    breaklines=true,                 
    captionpos=b,                    
    keepspaces=true,                 
    numbers=left,                    
    numbersep=5pt,                  
    showspaces=false,                
    showstringspaces=false,
    showtabs=false,                  
    tabsize=2
}

\lstset{style=mystyle}

% Page Geometry
\geometry{a4paper, margin=1in}

% Title Information
\title{Home Brew Advantage: The Gravitational Influence of Regional Coffee Chains on Super Bowl LX}
\author{Charlie Thompson \\ \textit{Prime Minister, Ministry of Silly Plots}}
\date{January 28, 2026}

\begin{document}

\maketitle

\begin{abstract}
This study investigates the correlation between the density of specific coffee chains (Starbucks vs. Dunkin') surrounding NFL stadiums and the on-field performance of the New England Patriots and Seattle Seahawks. By implementing specific experimental controls—most notably isolating \textbf{Away Games} to eliminate Home Field Advantage bias—we identify a striking and robust performance separation that aligns with the ``Home Brew Advantage'' hypothesis. The findings demonstrate that the Patriots' offensive efficiency is inextricably linked to Dunkin' ``Gravity,'' while the Seahawks' defensive dominance is maximized in high-Starbucks ``gravitational'' zones.
\end{abstract}

\section{Introduction}

For decades, pundits have analyzed conventional factors such as weather, turf type, and crowd noise. This paper proposes a novel environmental variable: the ``Regional Coffee Chain Gravitational Pull.'' We hypothesize that the regional dominance of major coffee chains exerts a measurable influence on team performance, specifically for teams with strong cultural associations to those brands.

We propose the following hypotheses:
\begin{itemize}
    \item \textbf{H1 (Patriots):} The Pats Run on Dunkin'.The New England Patriots offense performs significantly better in environments with high \textbf{Dunkin' Gravity} ($G_{net} > 0$).
    \item \textbf{H2 (Seahawks):} The Legion of Brew. The Seattle Seahawks defense performs significantly better in environments with high \textbf{Starbucks Gravity} ($G_{net} < 0$).
    \item \textbf{H3 (Null Hypothesis):} These performance deltas are solely artifacts of Home Field Advantage.
\end{itemize}

\section{Methodology}

\subsection{The Coffee Gravity Model}

To quantify the ``coffee environment'' of each stadium, we employed an \textbf{Interference-Adjusted Exponential Decay Model}. Conventional density metrics fail to account for proximity; a coffee shop adjacent to the stadium should carry more weight than one ten miles away.

The gravitational pull $G_{chain}$ for a given chain is calculated as:

\begin{equation}
    G_{chain} = \sum_{i=0}^{n} M_i \cdot e^{-0.5 \cdot d_i}
\end{equation}

Where:
\begin{itemize}
    \item $d_i$ is the Haversine distance (in miles) from the stadium to location $i$, defined as:
    \begin{equation}
        d = 2r \arcsin\left(\sqrt{\sin^2\left(\frac{\phi_2 - \phi_1}{2}\right) + \cos(\phi_1) \cos(\phi_2) \sin^2\left(\frac{\lambda_2 - \lambda_1}{2}\right)}\right)
    \end{equation}
    where $\phi$ represents latitude, $\lambda$ represents longitude, and $r$ is the radius of the Earth (approx. 3959 miles).
    \item $M_i$ is the ``Mass'' of location $i$, initialized at 1.0.
\end{itemize}

\subsubsection{Interference Term}
To account for market saturation and competition, we introduced an interference term. The mass $M_i$ is reduced if a competitor's location is within an \textbf{Interference Radius} ($r = 0.5$ miles). As direct competitors with a distinct regional split, we treat Starbucks and Dunkin' as mutually exclusive and assume that the presence of one negates the gravitational pull of the other.

\begin{equation}
    M_i' = M_i - \left(1.0 - \frac{d_{competitor}}{0.5}\right)
\end{equation}

The Net Gravity ($G_{net}$) is defined as the difference between the two forces:

\begin{equation}
    G_{net} = G_{dunkin} - G_{starbucks}
\end{equation}

Positive values indicate a Dunkin'-dominant environment, while negative values indicate a Starbucks-dominant environment.

\subsection{Experimental Controls}

Home field advantage is a clear confounding variable in this analysis, since teams play better at home in general and the two teams in question have particular strong gravitational fields for their respective coffee chains. To address this, we applied a strict \textbf{Away Games Only} filter. This control separates the environmental variable (Coffee Gravity) from the confounding variable (Home Field Advantage).

\subsection{Data Sources}

The analysis utilizes data from the 2025 NFL Season (Regular Season + Playoffs). Game data was sourced from \texttt{nflverse} Play-by-Play data via BigQuery, filtered for \texttt{season\_type IN ('REG', 'POST')}. Location data consists of geocoded coordinates of all US Starbucks and Dunkin' locations.

\begin{figure}[H]
    \centering
    \begin{subfigure}[b]{0.45\textwidth}
        \centering
        \includegraphics[width=\textwidth]{screenshots/New_England_Patriots_Gillette_Stadium.png}
        \caption{Gillette Stadium (High Dunkin' Density)}
        \label{fig:gillette}
    \end{subfigure}
    \hfill
    \begin{subfigure}[b]{0.45\textwidth}
        \centering
        \includegraphics[width=\textwidth]{screenshots/Seattle_Seahawks_Lumen_Field.png}
        \caption{Lumen Field (High Starbucks Density)}
        \label{fig:lumen}
    \end{subfigure}
    \caption{Visualizing the Coffee Gravity Field: Clean screenshots of stadium environments used for density calculation.}
    \label{fig:stadiums}
\end{figure}

\section{Results}

\begin{figure}[H]
    \centering
    \includegraphics[width=\textwidth]{coffee_gravity_ranking_publication.jpeg}
    \caption{The Home Brew Advantage: Net Coffee Gravity for all NFL teams. Stadiums below the dashed line have higher Dunkin' gravity; stadiums above the dashed line have higher Starbucks gravity.}
    \label{fig:ranking}
\end{figure}

\subsection{The Patriots ``Run on Dunkin'' (Confirmed)}

Analyzing \textbf{Away Games Only}, the Patriots offense shows a drastic drop in production when entering ``Starbucks Gravitational Zones'' ($G_{net} < 0$). This is primarily driven by a collapse in the rushing attack.

\begin{table}[H]
    \centering
    \caption{Patriots Offensive Metrics (Away Games Only)}
    \begin{tabular}{lccl}
        \toprule
        \textbf{Metric} & \textbf{Dunkin Zone} & \textbf{Starbucks Zone} & \textbf{Delta} \\
         & ($G_{net} > 0$) & ($G_{net} < 0$) & \\
        \midrule
        Points Per Game & 31.3 & 24.0 & \textbf{-7.3} \\
        Total Yds / Game & 409.7 & 338.5 & \textbf{-71.2} \\
        Rush EPA / Play & +0.053 & -0.186 & -0.239 \\
        \bottomrule
    \end{tabular}
\end{table}

\textbf{Conclusion:} H1 is \textbf{Empirically Validated}. The data reveals a binary mode of performance: in Dunkin' zones, the offense operates at an elite efficiency; in Starbucks zones, production collapses across all key vectors.

\subsection{The Seahawks ``Legion of Brew'' (Confirmed)}

Conversely, the Seahawks defense exhibits elite performance metrics in ``Starbucks Gravitational Zones.''

\begin{table}[H]
    \centering
    \caption{Seahawks Defensive Metrics (Away Games Only)}
    \begin{tabular}{lccl}
        \toprule
        \textbf{Metric} & \textbf{Dunkin Zone} & \textbf{Starbucks Zone} & \textbf{Delta} \\
         & ($G_{net} > 0$) & ($G_{net} < 0$) & \\
        \midrule
        Total Turnovers & 4 & 9 & +5 \\
        Turnovers / Game & 1.00 & \textbf{1.80} & \textbf{+0.80} \\
        PPG Allowed & 14.8 & 14.2 & -0.6 \\
        Opp. Passer Rtg & 70.3 & \textbf{61.6} & \textbf{-8.7} \\
        \bottomrule
    \end{tabular}
\end{table}

\textbf{Conclusion:} H2 is \textbf{Strongly Supported}. The Starbucks atmosphere correlates with a massive +80\% increase in forced turnovers (1.8 vs 1.0 per game), providing strong evidence for the environmental influence of the coffee ecosystem.

\subsection{Outlier Discovery: The Sam Darnold Paradox}

An unexpected finding emerged regarding Seahawks QB Sam Darnold. Unlike his team's defense, Darnold exhibits a strong \textbf{negative correlation} with Starbucks Gravity. While more research is needed to determine the cause of this phenomenon, one hypothesis is that Darnold is still seeing ghosts from his time on the East Coast with the Jets and still has an innate affinity for Dunkin'.

\begin{table}[H]
    \centering
    \caption{Sam Darnold Performance Splits}
    \begin{tabular}{lccc}
        \toprule
        \textbf{Metric} & \textbf{Dunkin Zone} & \textbf{Starbucks Zone} & \textbf{Delta} \\
        \midrule
        Passer Rating & \textbf{124.4} & 75.4 & \textbf{-49.0} \\
        TD / INT Ratio & \textbf{5.50} & 0.57 & \textbf{-4.93} \\
        Points Per Game & \textbf{31.2} & 22.6 & \textbf{-8.6} \\
        \bottomrule
    \end{tabular}
\end{table}

\textbf{Interpretation:} This finding is \textbf{Anomalous yet Distinct}. While the Seahawks \emph{Defense} thrives in Starbucks zones, their Quarterback appears to suffer an allergic reaction to the same environment (75.4 Rating vs 124.4 in Dunkin' zones). This suggests a fundamental biochemical incompatibility between Darnold and the Starbucks ecosystem.

\section{Discussion: Super Bowl LX Preview}

Super Bowl LX will be held at \textbf{Levi's Stadium} in Santa Clara, CA. Our model calculates the Net Gravity of this location to be \textbf{-5.80}, the second most Starbucks-dominant stadium in the league, behind only the Seahawks' own Lumen Field. It also has higher Starbucks Gravity than the Patriots' home stadium, Gillette Stadium has Dunkin' gravity. Assuming equal Dunkin' and Starbucks gravitational effects for both teams, the Patriots are at a significant disadvantage.

\begin{figure}[H]
    \centering
    \includegraphics[width=0.8\textwidth]{screenshots/San_Francisco_49ers_Levi's_Stadium.png}
    \caption{Levi's Stadium: A Starbucks stronghold ($G_{net} = -5.80$).}
    \label{fig:levis}
\end{figure}

Based on the robust ``Away Games Only'' model, we project the following outcomes:
\begin{enumerate}
    \item \textbf{Patriots Offense:} Predicted to underperform ($\approx$ 24 PPG, -0.186 rushing EPA/play).
    \item \textbf{Seahawks Defense:} Predicted to dominate ($\approx$ 12 PPG allowed, +80\% increase in takeaways).
    \item \textbf{Variable:} Sam Darnold is predicted to struggle significantly (75.4 Passer Rating), though the Seahawks defense is likely to slow down Drake Maye even more, holding opposing QBs to a 61.6 passer rating in Starbucks zones.
\end{enumerate}

\textbf{Prediction:} The environmental factors heavily favor a \textbf{Seahawks Defensive Victory}, provided their run game can compensate for the predicted Quarterback inefficiency.

\newpage
\appendix
\section{Codebase Appendix}

\subsection{Map Generation Script}
\begin{lstlisting}[language=Python, caption=generate\_gravity\_field\_all.py]
import os
import shutil
import pandas as pd
import numpy as np
import folium
import matplotlib.pyplot as plt
try:
    from google.cloud import bigquery
    from google.oauth2 import service_account
    BQ_AVAILABLE = True
except ImportError:
    print("Google Cloud libraries not found. Will use JSON cache.")
    BQ_AVAILABLE = False
except Exception as e:
    print(f"Error importing Google Cloud libraries: {e}")
    BQ_AVAILABLE = False

from math import radians, cos, sin, asin, sqrt

# --- Configuration ---
PROJECT_ROOT = os.path.abspath(os.path.join(os.path.dirname(__file__), '../../../../'))
BQ_COFFEE_TABLE = "stuperlatives.coffee_wars"
INTERFERENCE_RADIUS = 0.5
INTERFERENCE_STRENGTH = 1.0
GRID_RES = 100 # Lower res per stadium since we have many
OVERLAY_DIR = "map_overlays"

def get_bq_client():
    if not BQ_AVAILABLE:
        return None
    try:
        possible_keys = [
            'shhhh/service_account.json',
            '../../../shhhh/service_account.json',
            os.path.join(PROJECT_ROOT, 'shhhh/service_account.json')
        ]
        key_path = next((p for p in possible_keys if os.path.exists(p)), None)
        
        if key_path:
            credentials = service_account.Credentials.from_service_account_file(key_path)
            return bigquery.Client(credentials=credentials, project=credentials.project_id)
        else:
            return bigquery.Client()
    except Exception as e:
        print(f"Error creating BQ client: {e}")
        return None

def haversine_vectorized(lon1, lat1, lon2_array, lat2_array):
    lon1, lat1, lon2, lat2 = map(np.radians, [lon1, lat1, lon2_array, lat2_array])
    dlon = lon2 - lon1
    dlat = lat2 - lat1
    a = np.sin(dlat/2.0)**2 + np.cos(lat1) * np.cos(lat2) * np.sin(dlon/2.0)**2
    c = 2 * np.arcsin(np.sqrt(a))
    return 3956 * c

def simple_hav(lo1, la1, lo2, la2):
    lo1, la1, lo2, la2 = map(radians, [lo1, la1, lo2, la2])
    dlon = lo2 - lo1; dlat = la2 - la1
    a = sin(dlat/2)**2 + cos(la1)*cos(la2)*sin(dlon/2)**2
    return 2 * asin(sqrt(a)) * 3956

def generate_all_maps():
    client = get_bq_client()
    
    df = pd.DataFrame()
    if client:
        try:
            cols = ['team_name', 'stadium_name', 'dunkin', 'starbucks']
            query = f"""SELECT {','.join(cols)} FROM `{BQ_COFFEE_TABLE}`"""
            df = client.query(query).to_dataframe()
        except Exception as e:
            print(f"BQ Query failed: {e}")
    
    # If no DF from BQ, try loading from JSON cache
    if df.empty:
        # Try local path (same dir as script)
        json_path = os.path.join(os.path.dirname(__file__), "coffee_data_cache.json")
        if not os.path.exists(json_path):
             # Try alternate path just in case
             json_path = os.path.join(PROJECT_ROOT, "posts/super_bowl/data/prep/coffee_data_cache.json")

        if os.path.exists(json_path):
            print(f"Loading data from cache: {json_path}")
            import json
            with open(json_path, 'r') as f:
                data = json.load(f)
            
            # Convert to DataFrame with expected columns
            rows = []
            for entry in data:
                rows.append({
                    'team_name': entry['Team'],
                    'stadium_name': entry['Stadium'],
                    'dunkin': entry.get('Dunkin_Stats', {}),
                    'starbucks': entry.get('Starbucks_Stats', {})
                })
            df = pd.DataFrame(rows)
        else:
            print("No BQ data and no cache found.")
            return

    # Map Full Name to Abbrev
    TEAM_TO_ABBREV = {
        'Arizona Cardinals': 'ARI', 'Atlanta Falcons': 'ATL', 'Baltimore Ravens': 'BAL', 
        'Buffalo Bills': 'BUF', 'Carolina Panthers': 'CAR', 'Chicago Bears': 'CHI', 
        'Cincinnati Bengals': 'CIN', 'Cleveland Browns': 'CLE', 'Dallas Cowboys': 'DAL', 
        'Denver Broncos': 'DEN', 'Detroit Lions': 'DET', 'Green Bay Packers': 'GB', 
        'Houston Texans': 'HOU', 'Indianapolis Colts': 'IND', 'Jacksonville Jaguars': 'JAX', 
        'Kansas City Chiefs': 'KC', 'Las Vegas Raiders': 'LV', 'Los Angeles Chargers': 'LAC', 
        'Los Angeles Rams': 'LA', 'Miami Dolphins': 'MIA', 'Minnesota Vikings': 'MIN', 
        'New England Patriots': 'NE', 'New Orleans Saints': 'NO', 'New York Giants': 'NYG', 
        'New York Jets': 'NYJ', 'Philadelphia Eagles': 'PHI', 'Pittsburgh Steelers': 'PIT', 
        'San Francisco 49ers': 'SF', 'Seattle Seahawks': 'SEA', 'Tampa Bay Buccaneers': 'TB', 
        'Tennessee Titans': 'TEN', 'Washington Commanders': 'WAS'
    }

    # Reverse mapping for lookups
    TEAM_TO_ABBREV_REVERSE = {v: k for k, v in TEAM_TO_ABBREV.items()}

    # Team Primary Colors (for popup styling)
    TEAM_COLORS = {
        'ARI': '#97233F', 'ATL': '#A71930', 'BAL': '#241773', 'BUF': '#00338D',
        'CAR': '#0085CA', 'CHI': '#0B162A', 'CIN': '#FB4F14', 'CLE': '#311D00',
        'DAL': '#041E42', 'DEN': '#FB4F14', 'DET': '#0076B6', 'GB': '#203731',
        'HOU': '#03202F', 'IND': '#002C5F', 'JAX': '#006778', 'KC': '#E31837',
        'LV': '#000000', 'LAC': '#0080C6', 'LA': '#003594', 'MIA': '#008E97',
        'MIN': '#4F2683', 'NE': '#002244', 'NO': '#D3BC8D', 'NYG': '#0B2265',
        'NYJ': '#125740', 'PHI': '#004C54', 'PIT': '#FFB612', 'SF': '#AA0000',
        'SEA': '#002244', 'TB': '#D50A0A', 'TEN': '#0C2340', 'WAS': '#5A1414'
    }

    # Latest PBP Stadium Names (Must match BQ stuperlatives.coffee_wars 'stadium_name' EXACTLY)
    CURRENT_STADIUM_NAMES = {
        'NE': 'Gillette Stadium',
        'CHI': 'Soldier Field',
        'DEN': 'Empower Field at Mile High',
        'SEA': 'Lumen Field',
        'PIT': 'Acrisure Stadium',
        'PHI': 'Lincoln Financial Field',
        'JAX': 'EverBank Stadium',
        'CAR': 'Bank of America Stadium',
        'MIN': 'U.S. Bank Stadium',
        'LA': 'SoFi Stadium',
        'NYG': 'MetLife Stadium',
        'HOU': 'NRG Stadium',
        'ATL': 'Mercedes-Benz Stadium',
        'LV': 'Allegiant Stadium',
        'BUF': 'Highmark Stadium',
        'CIN': 'Paycor Stadium',
        'SF': "Levi's Stadium",          # BQ Name (Not Levi's®)
        'TB': 'Raymond James Stadium',
        'TEN': 'Nissan Stadium',
        'IND': 'Lucas Oil Stadium',
        'NYJ': 'MetLife Stadium',
        'CLE': 'Cleveland Browns Stadium', # BQ Name (Not Huntington)
        'MIA': 'Hard Rock Stadium',
        'GB': 'Lambeau Field',
        'LAC': 'SoFi Stadium',
        'KC': 'GEHA Field at Arrowhead Stadium',
        'WAS': 'Commanders Field',         # BQ Name (Not Northwest)
        'NO': 'Caesars Superdome',
        'ARI': 'State Farm Stadium',
        'DET': 'Ford Field',
        'DAL': 'AT&T Stadium',
        'BAL': 'M&T Bank Stadium',
    }

    # Coordinates (Keys must match BQ Names above)
    STADIUM_COORDS = {
        "State Farm Stadium": (33.5276, -112.2626),
        "Mercedes-Benz Stadium": (33.7554, -84.4010),
        "M&T Bank Stadium": (39.2780, -76.6227),
        "Highmark Stadium": (42.7738, -78.7870),
        "Bank of America Stadium": (35.2258, -80.8528),
        "Soldier Field": (41.8623, -87.6167),
        "Paycor Stadium": (39.0955, -84.5161),
        "Cleveland Browns Stadium": (41.5061, -81.6995),
        "AT&T Stadium": (32.7478, -97.0928),
        "Empower Field at Mile High": (39.7439, -105.0201),
        "Ford Field": (42.3400, -83.0456),
        "Lambeau Field": (44.5013, -88.0622),
        "NRG Stadium": (29.6847, -95.4107),
        "Lucas Oil Stadium": (39.7601, -86.1639),
        "EverBank Stadium": (30.3239, -81.6373),
        "GEHA Field at Arrowhead Stadium": (39.0489, -94.4839),
        "SoFi Stadium": (33.9535, -118.3390),
        "Allegiant Stadium": (36.0909, -115.1833),
        "Hard Rock Stadium": (25.9580, -80.2389),
        "U.S. Bank Stadium": (44.9739, -93.2581),
        "Gillette Stadium": (42.0909, -71.2643),
        "Caesars Superdome": (29.9511, -90.0812),
        "MetLife Stadium": (40.8135, -74.0745),
        "Lincoln Financial Field": (39.9008, -75.1675),
        "Acrisure Stadium": (40.4468, -80.0158),
        "Levi's Stadium": (37.4032, -121.9698),
        "Lumen Field": (47.5952, -122.3316),
        "Raymond James Stadium": (27.9759, -82.5033),
        "Nissan Stadium": (36.1664, -86.7713),
        "Commanders Field": (38.9076, -76.8645),
    }

    # Display Overrides (BQ Name -> Modern Display Name)
    DISPLAY_OVERRIDES = {
        "Cleveland Browns Stadium": "Huntington Bank Field",
        "Levi's Stadium": "Levi's® Stadium",
        "Commanders Field": "Northwest Stadium"
    }

    # Initialize Global Map (Center of US roughly)
    # UPDATED: Focus on Super Bowl LX (Levi's Stadium)
    m = folium.Map(location=[37.4032, -121.9698], zoom_start=11, tiles='CartoDB dark_matter')
    
    # Track processed teams
    processed_teams = set()

    # Iterate Stadiums
    for idx, row in df.iterrows():
        team_full = row['team_name']
        team_abbr = TEAM_TO_ABBREV.get(team_full)
        current_stadium_bq = row['stadium_name']
        
        if not team_abbr: continue 
        
        # 1. Identify what the BQ row SHOULD be for this team
        target_bq_name = CURRENT_STADIUM_NAMES.get(team_abbr)
        if not target_bq_name: continue
        
        # 2. Strict Match: Only process if the ROW matches the TARGET
        # This prevents 'Candlestick Park' row from being processed for SF
        if current_stadium_bq != target_bq_name:
            continue
            
        # 3. Avoid duplicates (e.g. NYG processed, then find NYG again?)
        if team_full in processed_teams: continue
        
        # Coordinates
        if target_bq_name not in STADIUM_COORDS:
            print(f"Warning: No coords for {target_bq_name}")
            continue
        stadium_lat, stadium_lng = STADIUM_COORDS[target_bq_name]
        
        processed_teams.add(team_full)
        
        # Display Name
        stadium_display_name = DISPLAY_OVERRIDES.get(target_bq_name, target_bq_name)
        
        print(f"Processing {team_full} -> {target_bq_name} (Display: {stadium_display_name})...")
        
        d_locs = row['dunkin'].get('locations', []) if row['dunkin'] else []
        s_locs = row['starbucks'].get('locations', []) if row['starbucks'] else []
        if d_locs is None: d_locs = []
        if s_locs is None: s_locs = []
        
        if len(d_locs) == 0 and len(s_locs) == 0: continue
        
        # 1. Calc Masses
        d_masses = np.ones(len(d_locs))
        s_masses = np.ones(len(s_locs))
        
        if len(d_locs) > 0 and len(s_locs) > 0:
            for i, d in enumerate(d_locs):
                for j, S in enumerate(s_locs):
                    dist = simple_hav(d['lng'], d['lat'], S['lng'], S['lat'])
                    if dist < INTERFERENCE_RADIUS:
                        reduction = INTERFERENCE_STRENGTH * (1.0 - dist/INTERFERENCE_RADIUS)
                        d_masses[i] -= reduction
                        s_masses[j] -= reduction
            d_masses = np.maximum(d_masses, 0.0)
            s_masses = np.maximum(s_masses, 0.0)

        # 2. Local Grid
        # Center grid on the ACTUAL stadium coordinates
        center_lat = stadium_lat
        center_lng = stadium_lng
        
        # Determine bounds based on actual store spread + generous fade-out buffer
        # But ensure we respect the stadium center so the view is nice.
        # Actually, let's use the store spread if it's wider, but center on the stadium.
        
        # Collect store coords for sizing
        all_lats = [x['lat'] for x in d_locs] + [x['lat'] for x in s_locs]
        all_lngs = [x['lng'] for x in d_locs] + [x['lng'] for x in s_locs]
        
        # If stores are far, extend bounds.
        # If stores are close, enforce minimum bound around stadium.
        
        buffer = 0.22 # Approx 15 miles
        
        min_lat = min(min(all_lats), center_lat) - buffer
        max_lat = max(max(all_lats), center_lat) + buffer
        
        # Longitude correction
        lng_buffer = buffer / np.cos(np.radians(center_lat))
        min_lng = min(min(all_lngs), center_lng) - lng_buffer
        max_lng = max(max(all_lngs), center_lng) + lng_buffer
        
        lats = np.linspace(min_lat, max_lat, GRID_RES)
        lngs = np.linspace(min_lng, max_lng, GRID_RES)
        lng_mesh, lat_mesh = np.meshgrid(lngs, lats)
        
        # 3. Field Calc
        d_field = np.zeros(lng_mesh.shape)
        s_field = np.zeros(lng_mesh.shape)
        
        for i, loc in enumerate(d_locs):
            if d_masses[i] > 0:
                dist = haversine_vectorized(loc['lng'], loc['lat'], lng_mesh, lat_mesh)
                d_field += d_masses[i] * np.exp(-0.5 * dist)
                
        for i, loc in enumerate(s_locs):
            if s_masses[i] > 0:
                dist = haversine_vectorized(loc['lng'], loc['lat'], lng_mesh, lat_mesh)
                s_field += s_masses[i] * np.exp(-0.5 * dist)
        
        # 4. Net Gravity Coloring (Stark Contrast)
        # Net Gravity defined as sum(D * e^-0.5d) - sum(S * e^-0.5d)
        net_field = d_field - s_field
        
        # Normalize
        limit = max(np.abs(np.min(net_field)), np.abs(np.max(net_field)))
        if limit == 0: limit = 1
        
        # Create RGBA array
        rgba = np.zeros((GRID_RES, GRID_RES, 4))
        
        # Mask for positive (Dunkin)
        pos_mask = net_field > 0
        strength = np.abs(net_field) / limit
        
        # Dunkin Orange: #FF671F -> (1.0, 0.4, 0.12)
        rgba[pos_mask, 0] = 1.0
        rgba[pos_mask, 1] = 0.4
        rgba[pos_mask, 2] = 0.12
        rgba[pos_mask, 3] = strength[pos_mask] * 1.0 # Max Stark Opacity
        
        # Starbucks Green: #00704A -> (0.0, 0.44, 0.29)
        neg_mask = net_field < 0
        rgba[neg_mask, 0] = 0.0
        rgba[neg_mask, 1] = 0.44
        rgba[neg_mask, 2] = 0.29
        rgba[neg_mask, 3] = strength[neg_mask] * 1.0 # Max Stark Opacity
        
        # Flip for Map
        rgba = np.flipud(rgba)
        
        # Save Overlay
        if not os.path.exists(OVERLAY_DIR):
            os.makedirs(OVERLAY_DIR)
        
        safe_name = "".join(x for x in team_full if x.isalnum()) + "_" + "".join(x for x in stadium_display_name if x.isalnum())
        img_name = f"{OVERLAY_DIR}/{safe_name}.png"
        plt.imsave(img_name, rgba)
        
        # Add to Map
        folium.raster_layers.ImageOverlay(
            image=img_name,
            bounds=[[min_lat, min_lng], [max_lat, max_lng]],
            opacity=0.9,
            interactive=True,
            cross_origin=False,
            zindex=1
        ).add_to(m)
        
        # Store Markers (Detailed)
        # Only plot if significant mass
        for i, loc in enumerate(d_locs):
            if d_masses[i] > 0.1: # Threshold to reduce clutter
                folium.CircleMarker(
                    [loc['lat'], loc['lng']], radius=3, color='gray', weight=0.5, fill=True, fill_color='#FF671F', fill_opacity=1.0
                ).add_to(m)
                
        for i, loc in enumerate(s_locs):
            if s_masses[i] > 0.1:
                folium.CircleMarker(
                    [loc['lat'], loc['lng']], radius=3, color='gray', weight=0.5, fill=True, fill_color='#00704A', fill_opacity=1.0
                ).add_to(m)
        
        
        # Calculate Net Gravity at Stadium Location
        # Sum all store contributions at the stadium coordinates
        stadium_d_gravity = 0
        stadium_s_gravity = 0
        
        for i, loc in enumerate(d_locs):
            if d_masses[i] > 0:
                dist = simple_hav(loc['lng'], loc['lat'], center_lng, center_lat)
                stadium_d_gravity += d_masses[i] * np.exp(-0.5 * dist)
        
        for i, loc in enumerate(s_locs):
            if s_masses[i] > 0:
                dist = simple_hav(loc['lng'], loc['lat'], center_lng, center_lat)
                stadium_s_gravity += s_masses[i] * np.exp(-0.5 * dist)
        
        net_gravity = stadium_d_gravity - stadium_s_gravity
        
        # Determine winner (no ties - always pick the higher side)
        if net_gravity >= 0:
            winner = "Dunkin'"
            winner_color = "#FF671F"
            winner_logo = "logos/dunkin.png"
        else:
            winner = "Starbucks"
            winner_color = "#00704A"
            winner_logo = "logos/starbucks.png"
        
        # Coffee chain logos for the breakdown
        dunkin_logo = "logos/dunkin.png"
        starbucks_logo = "logos/starbucks.png"
        
        # Marker with Team Logo(s)
        # ESPN logo URL format: https://a.espncdn.com/i/teamlogos/nfl/500/{team_abbr}.png
        
        # For teams sharing a stadium (NYG/NYJ, LA/LAC), place logos side-by-side
        teams_at_stadium = [t for t, s in CURRENT_STADIUM_NAMES.items() if s == target_bq_name]
        
        if len(teams_at_stadium) > 1:
            # Multiple teams - offset logos horizontally
            # Offset by ~0.002 degrees lng (~200m at mid-latitudes)
            offset = 0.002
            num_teams = len(teams_at_stadium)
            
            # Center the group of logos
            start_offset = -offset * (num_teams - 1) / 2
            
            for idx, team_abbr_shared in enumerate(teams_at_stadium):
                team_full_shared = TEAM_TO_ABBREV_REVERSE.get(team_abbr_shared, team_abbr_shared)
                logo_url = f"https://a.espncdn.com/i/teamlogos/nfl/500/{team_abbr_shared}.png"
                offset_lng = center_lng + start_offset + (idx * offset)
                team_color = TEAM_COLORS.get(team_abbr_shared, '#1e3c72')
                
                # Create combined team name for shared stadiums
                if len(teams_at_stadium) > 1:
                    all_team_names = [TEAM_TO_ABBREV_REVERSE.get(t, t) for t in teams_at_stadium]
                    team_name_display = " / ".join(all_team_names)
                else:
                    team_name_display = team_full_shared
                
                # Styled popup
                abs_gravity = abs(net_gravity)
                popup_html = f"""
                <div style="font-family: 'Arial', sans-serif; min-width: 220px;">
                    <div style="background: linear-gradient(135deg, {team_color} 0%, {team_color}dd 100%); 
                                color: white; padding: 12px; margin: -10px -10px 10px -10px; 
                                border-radius: 4px 4px 0 0; text-align: center;">
                        <div style="font-size: 18px; font-weight: bold; margin-bottom: 4px;">{stadium_display_name}</div>
                        <div style="font-size: 12px; opacity: 0.9;">{team_name_display}</div>
                    </div>
                    <div style="padding: 8px 4px;">
                        <div style="margin-bottom: 12px; text-align: center; padding: 10px; background: #f5f5f5; border-radius: 4px;">
                            <div style="font-weight: bold; font-size: 12px; color: #666; margin-bottom: 4px;">Net Gravity</div>
                            <div style="font-size: 22px; font-weight: bold; color: {winner_color};">
                                +{abs_gravity:.2f}
                                <img src="{winner_logo}" style="height: 24px; vertical-align: middle; margin-left: 8px;">
                            </div>
                        </div>
                        <div style="font-size: 13px; color: #666;">
                            <div style="display: flex; align-items: center; margin-bottom: 6px; padding: 4px; background: #fff3e6; border-radius: 3px;">
                                <img src="{dunkin_logo}" style="height: 18px; margin-right: 6px;">
                                <span style="margin-left: auto; font-weight: bold; color: #FF671F;">{stadium_d_gravity:.2f}</span>
                            </div>
                            <div style="display: flex; align-items: center; margin-bottom: 6px; padding: 4px; background: #e6f5f0; border-radius: 3px;">
                                <img src="{starbucks_logo}" style="height: 18px; margin-right: 6px;">
                                <span style="margin-left: auto; font-weight: bold; color: #00704A;">{stadium_s_gravity:.2f}</span>
                            </div>
                        </div>
                    </div>
                </div>
                """
                
                folium.Marker(
                    location=[center_lat, offset_lng],
                    popup=folium.Popup(popup_html, max_width=250),
                    icon=folium.CustomIcon(logo_url, icon_size=(30, 30))
                ).add_to(m)
        else:
            # Single team
            logo_url = f"https://a.espncdn.com/i/teamlogos/nfl/500/{team_abbr}.png"
            team_color = TEAM_COLORS.get(team_abbr, '#1e3c72')
            
            # Styled popup
            abs_gravity = abs(net_gravity)
            popup_html = f"""
            <div style="font-family: 'Arial', sans-serif; min-width: 220px;">
                <div style="background: linear-gradient(135deg, {team_color} 0%, {team_color}dd 100%); 
                            color: white; padding: 12px; margin: -10px -10px 10px -10px; 
                            border-radius: 4px 4px 0 0; text-align: center;">
                    <div style="font-size: 18px; font-weight: bold; margin-bottom: 4px;">{stadium_display_name}</div>
                    <div style="font-size: 12px; opacity: 0.9;">{team_full}</div>
                </div>
                <div style="padding: 8px 4px;">
                    <div style="margin-bottom: 12px; text-align: center; padding: 10px; background: #f5f5f5; border-radius: 4px;">
                        <div style="font-weight: bold; font-size: 12px; color: #666; margin-bottom: 4px;">Net Gravity</div>
                        <div style="font-size: 22px; font-weight: bold; color: {winner_color};">
                            +{abs_gravity:.2f}
                            <img src="{winner_logo}" style="height: 24px; vertical-align: middle; margin-left: 8px;">
                        </div>
                    </div>
                    <div style="font-size: 13px; color: #666;">
                        <div style="display: flex; align-items: center; margin-bottom: 6px; padding: 4px; background: #fff3e6; border-radius: 3px;">
                            <img src="{dunkin_logo}" style="height: 18px; margin-right: 6px;">
                            <span style="margin-left: auto; font-weight: bold; color: #FF671F;">{stadium_d_gravity:.2f}</span>
                        </div>
                        <div style="display: flex; align-items: center; margin-bottom: 6px; padding: 4px; background: #e6f5f0; border-radius: 3px;">
                            <img src="{starbucks_logo}" style="height: 18px; margin-right: 6px;">
                            <span style="margin-left: auto; font-weight: bold; color: #00704A;">{stadium_s_gravity:.2f}</span>
                        </div>
                    </div>
                </div>
            </div>
            """
            
            # Show Levi's Stadium popup by default for Super Bowl preview
            folium.Marker(
                location=[center_lat, center_lng],
                popup=folium.Popup(popup_html, max_width=250, show=(target_bq_name == "Levi's Stadium")),
                icon=folium.CustomIcon(logo_url, icon_size=(30, 30))
            ).add_to(m)

    # ADD LEGEND
    legend_html = '''
    <div style="position: fixed; 
         bottom: 30px; right: 30px; width: 210px; height: 160px; 
         border: 1px solid #ddd; z-index:9999; font-size:13px;
         background-color: white; opacity: 0.95;
         border-radius: 8px; padding: 12px; box-shadow: 0 2px 6px rgba(0,0,0,0.2); font-family: sans-serif;">
         <div style="font-weight: bold; margin-bottom: 10px; font-size: 15px; border-bottom: 1px solid #eee; padding-bottom: 5px;">Coffee Gravity Legend</div>
         
         <div style="margin-bottom: 6px;">
            <div style="background:#FF671F; width:8px; height:8px; display: inline-block; vertical-align: middle; margin-right: 8px; border-radius: 50%;"></div>
            <span style="vertical-align: middle;">Dunkin' Location</span>
         </div>
         <div style="margin-bottom: 10px;">
            <div style="background:#00704A; width:8px; height:8px; display: inline-block; vertical-align: middle; margin-right: 8px; border-radius: 50%;"></div>
            <span style="vertical-align: middle;">Starbucks Location</span>
         </div>

         <div style="margin-bottom: 6px;">
            <div style="background:radial-gradient(circle, rgba(255, 103, 31, 0.9) 0%, rgba(255, 103, 31, 0.4) 100%); width:20px; height:20px; display: inline-block; vertical-align: middle; margin-right: 8px; border-radius: 50%;"></div>
            <span style="vertical-align: middle;">Net Dunkin' Force Field</span>
         </div>
         <div style="margin-bottom: 10px;">
            <div style="background:radial-gradient(circle, rgba(0, 112, 74, 0.9) 0%, rgba(0, 112, 74, 0.4) 100%); width:20px; height:20px; display: inline-block; vertical-align: middle; margin-right: 8px; border-radius: 50%;"></div>
            <span style="vertical-align: middle;">Net Starbucks Force Field</span>
         </div>
    </div>
    '''
    m.get_root().html.add_child(folium.Element(legend_html))

    # Save directly to assets folder to avoid duplication
    # Script is in data/prep/, assets is in ../../assets/
    output_dir = os.path.join(PROJECT_ROOT, "posts/super_bowl/assets")
    if not os.path.exists(output_dir):
        os.makedirs(output_dir)
        
    output_path = os.path.join(output_dir, "coffee_force_field_map_all.html")
    m.save(output_path)
    print(f"All-Stadium Map saved to {output_path}")

if __name__ == "__main__":
    generate_all_maps()
\end{lstlisting}

\subsection{Data Extraction & Analysis Script}
\begin{lstlisting}[language=Python, caption=robust\_coffee\_check.py]
import os
import sys
import pandas as pd
import numpy as np
from google.cloud import bigquery
from google.oauth2 import service_account

# Add project root to path
sys.path.append(os.path.abspath(os.path.join(os.path.dirname(__file__), '../../../')))

# --- Configuration ---
TARGET_SEASON = 2025
BQ_COFFEE_TABLE = "stuperlatives.coffee_wars"
BQ_PBP_TABLE = "stuperlatives.pbp_data"
PROJECT_ROOT = os.path.abspath(os.path.join(os.path.dirname(__file__), '../../../'))

def get_bq_client():
    """Get BigQuery client with service account credentials"""
    try:
        possible_keys = [
            'shhhh/service_account.json',
            '../../../shhhh/service_account.json',
            os.path.join(PROJECT_ROOT, 'shhhh/service_account.json')
        ]
        key_path = next((p for p in possible_keys if os.path.exists(p)), None)
        
        if key_path:
            credentials = service_account.Credentials.from_service_account_file(key_path)
            return bigquery.Client(credentials=credentials, project=credentials.project_id)
        else:
            return bigquery.Client()
    except Exception as e:
        print(f"Error creating BQ client: {e}")
        return None

def simple_hav(lo1, la1, lo2, la2):
    """Haversine distance in miles"""
    from math import radians, sin, cos, asin, sqrt
    lo1, la1, lo2, la2 = map(radians, [lo1, la1, lo2, la2])
    dlon = lo2 - lo1
    dlat = la2 - la1
    a = sin(dlat/2)**2 + cos(la1) * cos(la2) * sin(dlon/2)**2
    return 2 * asin(sqrt(a)) * 3956

def calculate_stadium_gravity_single(stadium_lat, stadium_lng, d_locs, s_locs):
    """
    Calculate net gravity at a stadium using the interference model
    Returns: (dunkin_gravity, starbucks_gravity, net_gravity)
    """
    INTERFERENCE_RADIUS = 0.5  # miles
    INTERFERENCE_STRENGTH = 1.0
    
    if len(d_locs) == 0 and len(s_locs) == 0:
        return 0.0, 0.0, 0.0
    
    # Initialize masses
    d_masses = np.ones(len(d_locs))
    s_masses = np.ones(len(s_locs))
    
    # Apply interference reduction
    if len(d_locs) > 0 and len(s_locs) > 0:
        for i, d in enumerate(d_locs):
            for j, s in enumerate(s_locs):
                dist = simple_hav(d['lng'], d['lat'], s['lng'], s['lat'])
                if dist < INTERFERENCE_RADIUS:
                    reduction = INTERFERENCE_STRENGTH * (1.0 - dist/INTERFERENCE_RADIUS)
                    d_masses[i] -= reduction
                    s_masses[j] -= reduction
        d_masses = np.maximum(d_masses, 0.0)
        s_masses = np.maximum(s_masses, 0.0)
    
    # Calculate gravity at stadium location
    dunkin_gravity = 0.0
    starbucks_gravity = 0.0
    
    for i, loc in enumerate(d_locs):
        if d_masses[i] > 0:
            dist = simple_hav(loc['lng'], loc['lat'], stadium_lng, stadium_lat)
            dunkin_gravity += d_masses[i] * np.exp(-0.5 * dist)
    
    for i, loc in enumerate(s_locs):
        if s_masses[i] > 0:
            dist = simple_hav(loc['lng'], loc['lat'], stadium_lng, stadium_lat)
            starbucks_gravity += s_masses[i] * np.exp(-0.5 * dist)
    
    net_gravity = dunkin_gravity - starbucks_gravity
    return dunkin_gravity, starbucks_gravity, net_gravity

def calculate_coffee_gravity(client):
    """
    Calculate Starbucks gravity for each stadium using the same model 
    Returns a DataFrame mapping home_team to net_gravity
    """
    print("Calculating Coffee Gravity for all stadiums...")
    
    # Stadium coordinates (current NFL stadiums)
    STADIUM_COORDS = {
        "State Farm Stadium": (33.5276, -112.2626),
        "Mercedes-Benz Stadium": (33.7554, -84.4010),
        "M&T Bank Stadium": (39.2780, -76.6227),
        "Highmark Stadium": (42.7738, -78.7870),
        "Bank of America Stadium": (35.2258, -80.8528),
        "Soldier Field": (41.8623, -87.6167),
        "Paycor Stadium": (39.0955, -84.5161),
        "Cleveland Browns Stadium": (41.5061, -81.6995),
        "AT&T Stadium": (32.7478, -97.0928),
        "Empower Field at Mile High": (39.7439, -105.0201),
        "Ford Field": (42.3400, -83.0456),
        "Lambeau Field": (44.5013, -88.0622),
        "NRG Stadium": (29.6847, -95.4107),
        "Lucas Oil Stadium": (39.7601, -86.1639),
        "EverBank Stadium": (30.3239, -81.6373),
        "GEHA Field at Arrowhead Stadium": (39.0489, -94.4839),
        "SoFi Stadium": (33.9535, -118.3390),
        "Allegiant Stadium": (36.0909, -115.1833),
        "Hard Rock Stadium": (25.9580, -80.2389),
        "U.S. Bank Stadium": (44.9739, -93.2581),
        "Gillette Stadium": (42.0909, -71.2643),
        "Caesars Superdome": (29.9511, -90.0812),
        "MetLife Stadium": (40.8135, -74.0745),
        "Lincoln Financial Field": (39.9008, -75.1675),
        "Acrisure Stadium": (40.4468, -80.0158),
        "Levi's Stadium": (37.4032, -121.9698),
        "Lumen Field": (47.5952, -122.3316),
        "Raymond James Stadium": (27.9759, -82.5033),
        "Nissan Stadium": (36.1664, -86.7713),
        "Commanders Field": (38.9076, -76.8645),
    }
    
    # Load coffee data from BigQuery
    query = f"""
    SELECT 
        stadium_name,
        team_name,
        dunkin,
        starbucks
    FROM `{BQ_COFFEE_TABLE}`
    """
    
    df = client.query(query).to_dataframe()
    
    # Calculate gravity for all current stadiums
    results = []
    for _, row in df.iterrows():
        stadium_name = row['stadium_name']
        team = row['team_name']
        
        # Only process current stadiums
        if stadium_name not in STADIUM_COORDS:
            continue
            
        stadium_lat, stadium_lng = STADIUM_COORDS[stadium_name]
        
        # Locations are in struct format from BigQuery
        d_locs = row['dunkin'].get('locations', []) if row['dunkin'] else []
        s_locs = row['starbucks'].get('locations', []) if row['starbucks'] else []
        
        # Convert to lists if needed
        if d_locs is None or (hasattr(d_locs, '__len__') and len(d_locs) == 0):
            d_locs = []
        if s_locs is None or (hasattr(s_locs, '__len__') and len(s_locs) == 0):
            s_locs = []
        
        d_grav, s_grav, net_grav = calculate_stadium_gravity_single(
            stadium_lat, stadium_lng, d_locs, s_locs
        )
        
        results.append({
            'stadium_name': stadium_name,
            'team_name': team,
            'dunkin_gravity': d_grav,
            'starbucks_gravity': s_grav,
            'net_gravity': net_grav,
        })
    
    gravity_df = pd.DataFrame(results)
    return gravity_df

def load_pbp_data(client, gravity_df):
    """Load PBP data and join with gravity"""
    print("Loading PBP Data...")
    
    query = f"""
    SELECT 
        season, week, game_id, 
        home_team, away_team, posteam, defteam,
        play_type, yards_gained, epa,
        pass_attempt, complete_pass, pass_touchdown, interception, sack,
        rush_attempt, rush_touchdown, fumble_lost,
        home_score, away_score,
        passer_player_name
    FROM `{BQ_PBP_TABLE}`
    WHERE season = {TARGET_SEASON}
    AND season_type IN ('REG', 'POST')
    AND play_type IN ('run', 'pass', 'no_play')
    """
    
    pbp_df = client.query(query).to_dataframe()
    
    # Map team names in gravity_df to abbr
    team_mapping = {
        'Arizona Cardinals': 'ARI', 'Atlanta Falcons': 'ATL', 'Baltimore Ravens': 'BAL',
        'Buffalo Bills': 'BUF', 'Carolina Panthers': 'CAR', 'Chicago Bears': 'CHI',
        'Cincinnati Bengals': 'CIN', 'Cleveland Browns': 'CLE', 'Dallas Cowboys': 'DAL',
        'Denver Broncos': 'DEN', 'Detroit Lions': 'DET', 'Green Bay Packers': 'GB',
        'Houston Texans': 'HOU', 'Indianapolis Colts': 'IND', 'Jacksonville Jaguars': 'JAX',
        'Kansas City Chiefs': 'KC', 'Los Angeles Chargers': 'LAC', 'San Diego Chargers': 'SD',
        'Los Angeles Rams': 'LA', 'St. Louis Rams': 'STL', 'Las Vegas Raiders': 'LV',
        'Oakland Raiders': 'OAK', 'Miami Dolphins': 'MIA', 'Minnesota Vikings': 'MIN',
        'New England Patriots': 'NE', 'New Orleans Saints': 'NO', 'New York Giants': 'NYG',
        'New York Jets': 'NYJ', 'Philadelphia Eagles': 'PHI', 'Pittsburgh Steelers': 'PIT',
        'San Francisco 49ers': 'SF', 'Seattle Seahawks': 'SEA', 'Tampa Bay Buccaneers': 'TB',
        'Tennessee Titans': 'TEN', 'Washington Commanders': 'WAS'
    }
    
    gravity_df['home_team_abbr'] = gravity_df['team_name'].map(team_mapping)
    
    # Join PBP with Gravity (on home_team)
    # Using left join to keep games even if stadium mapping fails (though it shouldn't for modern)
    merged = pd.merge(pbp_df, gravity_df, left_on='home_team', right_on='home_team_abbr', how='left')
    
    return merged

def calculate_metrics(df, team, is_offense=True, filter_desc=""):
    """
    Calculate the metrics for a subset of data.
    """
    metrics = {}
    
    # Filter for the specific team
    if is_offense:
        team_df = df[df['posteam'] == team].copy()
    else:
        team_df = df[df['defteam'] == team].copy()
        
    if len(team_df) == 0:
        return {k: 0 for k in ['Games', 'Comp %', 'YPA', 'TD/INT', 'Rating', 'Sack Rate', 'PPG', 'YPG', 'YPP', 'Rush EPA', 'Turnovers']}
    
    games = team_df['game_id'].nunique()
    metrics['Games'] = games
    
    if is_offense:
        # --- QB/Passing Metrics ---
        attempts = team_df['pass_attempt'].sum()
        completions = team_df['complete_pass'].sum()
        metrics['Comp %'] = (completions / attempts * 100) if attempts > 0 else 0.0
        
        pass_plays = team_df[(team_df['play_type'] == 'pass') & (team_df['sack'] == 0)]
        clean_pass_yards = pass_plays['yards_gained'].sum()
        metrics['YPA'] = (clean_pass_yards / attempts) if attempts > 0 else 0.0
        
        tds = team_df['pass_touchdown'].sum()
        ints = team_df['interception'].sum()
        metrics['TD/INT'] = (tds / ints) if ints > 0 else float(tds) # Or Inf
        
        if attempts > 0:
            a = (completions / attempts - 0.3) * 5
            b = (clean_pass_yards / attempts - 3) * 0.25
            c = (tds / attempts) * 20
            d = 2.375 - (ints / attempts * 25)
            rating = (max(0, min(2.375, a)) + max(0, min(2.375, b)) + max(0, min(2.375, c)) + max(0, min(2.375, d))) / 6 * 100
            metrics['Rating'] = rating
        else:
            metrics['Rating'] = 0.0
            
        sacks = team_df['sack'].sum()
        dropbacks = attempts + sacks
        metrics['Sack Rate'] = (sacks / dropbacks * 100) if dropbacks > 0 else 0.0
        
        game_scores = []
        for gid in team_df['game_id'].unique():
            g = team_df[team_df['game_id'] == gid].iloc[0]
            if g['home_team'] == team:
                game_scores.append(g['home_score'])
            else:
                game_scores.append(g['away_score'])
        metrics['PPG'] = np.mean(game_scores) if game_scores else 0.0
        
        total_yards = team_df['yards_gained'].sum()
        metrics['YPG'] = (total_yards / games) if games > 0 else 0.0
        
        plays = len(team_df[team_df['play_type'].isin(['run', 'pass'])])
        metrics['YPP'] = (total_yards / plays) if plays > 0 else 0.0
        
        rush_plays = team_df[team_df['play_type'] == 'run']
        rush_epa = rush_plays['epa'].mean()
        metrics['Rush EPA'] = rush_epa if not np.isnan(rush_epa) else 0.0
        
    else:
        # --- Defense Metrics ---
        game_scores_allowed = []
        for gid in team_df['game_id'].unique():
            g = team_df[team_df['game_id'] == gid].iloc[0]
            if g['home_team'] == team:
                game_scores_allowed.append(g['away_score']) # Allowed = opponent score
            else:
                game_scores_allowed.append(g['home_score'])
        metrics['PPG Allowed'] = np.mean(game_scores_allowed) if game_scores_allowed else 0.0
        
        opp_attempts = team_df['pass_attempt'].sum()
        opp_completions = team_df['complete_pass'].sum()
        opp_pass_plays = team_df[(team_df['play_type'] == 'pass') & (team_df['sack'] == 0)]
        opp_pass_yards = opp_pass_plays['yards_gained'].sum()
        
        opp_tds = team_df['pass_touchdown'].sum()
        opp_ints = team_df['interception'].sum()
        
        if opp_attempts > 0:
            a = (opp_completions / opp_attempts - 0.3) * 5
            b = (opp_pass_yards / opp_attempts - 3) * 0.25
            c = (opp_tds / opp_attempts) * 20
            d = 2.375 - (opp_ints / opp_attempts * 25)
            rating = (max(0, min(2.375, a)) + max(0, min(2.375, b)) + max(0, min(2.375, c)) + max(0, min(2.375, d))) / 6 * 100
            metrics['Opp Passer Rating'] = rating
        else:
            metrics['Opp Passer Rating'] = 0.0
        
        my_sacks = team_df['sack'].sum()
        opp_dropbacks = opp_attempts + my_sacks
        metrics['Sack Rate (Def)'] = (my_sacks / opp_dropbacks * 100) if opp_dropbacks > 0 else 0.0
        
        rush_plays = team_df[team_df['play_type'] == 'run']
        rush_yards_allowed = rush_plays['yards_gained'].sum()
        rush_attempts_against = team_df['rush_attempt'].sum()
        metrics['Opp YPC'] = (rush_yards_allowed / rush_attempts_against) if rush_attempts_against > 0 else 0.0

        ints = team_df['interception'].sum()
        fumbles = team_df['fumble_lost'].sum()
        total_turnovers = ints + fumbles
        metrics['Turnovers'] = total_turnovers
        metrics['Turnovers/Game'] = (total_turnovers / games) if games > 0 else 0.0

    return metrics

def main():
    client = get_bq_client()
    if not client: return

    # 1. Get Coffee Gravity
    gravity_df = calculate_coffee_gravity(client)
    
    # 2. Get PBP Data merged with Gravity
    df = load_pbp_data(client, gravity_df)
    
    # 3. Create subsets
    print("\n--- PATRIOTS (Runs on Dunkin) ---")
    ne_away = df[(df['posteam'] == 'NE') & (df['home_team'] != 'NE')]
    ne_dunkin = ne_away[ne_away['net_gravity'] > 0]
    ne_starbucks = ne_away[ne_away['net_gravity'] <= 0]
    
    print("\n--- SEAHAWKS (Chaos on Starbucks) ---")
    sea_away_def = df[(df['defteam'] == 'SEA') & (df['home_team'] != 'SEA')]
    sea_sb = sea_away_def[sea_away_def['net_gravity'] < 0]
    
    # Output logic ...
    
if __name__ == "__main__":
    main()
\end{lstlisting}

\subsection{SQL Reproduction Script}
\begin{lstlisting}[language=SQL, caption=reproduce\_robust\_metrics.sql]
/*
Reproducible SQL Queries for "Robust Coffee Metrics" Analysis
Target Season: 2025 (Regular + Playoffs)
METHODOLOGY:
1.  Gravity Calculation: Exponential decay model (gravity ~ exp(-0.5 * distance)) with interference.
2.  Filters: "Away Games Only" to remove home field advantage bias.
3.  Zones: Starbucks (<0), Dunkin (>0), Death Zone (<-4)
*/

-- ------------------------------------------------------------------
-- CTE: Stadium Gravity Mapping (Pre-calculated via Python model)
-- ------------------------------------------------------------------
WITH stadium_gravity AS (
    SELECT * FROM UNNEST([
        STRUCT('Empower Field at Mile High' AS stadium_name, 'Denver Broncos' AS team_name, -1.9701 AS net_gravity),
        STRUCT('GEHA Field at Arrowhead Stadium', 'Kansas City Chiefs', 0.0000),
        STRUCT('Ford Field', 'Detroit Lions', 2.3793),
        STRUCT('Lambeau Field', 'Green Bay Packers', 0.0000),
        STRUCT('Lumen Field', 'Seattle Seahawks', -11.4589),
        STRUCT('Levi\\'s Stadium', 'San Francisco 49ers', -5.7955),
        STRUCT('AT&T Stadium', 'Dallas Cowboys', 0.8654),
        -- ... (Truncated for brevity, full list in source)
        STRUCT('Lincoln Financial Field', 'Philadelphia Eagles', 0.9690)
    ])
),

-- ------------------------------------------------------------------
-- CTE: Team Abbreviation Mapping (Normalizing names)
-- ------------------------------------------------------------------
team_mapping AS (
    SELECT 'Denver Broncos' as name, 'DEN' as abbr UNION ALL
    SELECT 'Kansas City Chiefs', 'KC' UNION ALL
    SELECT 'Detroit Lions', 'DET' UNION ALL
    -- ... (Full list in source)
    SELECT 'Philadelphia Eagles', 'PHI'
),

-- ------------------------------------------------------------------
-- CTE: Joined PBP Data with Gravity (Away Games Only)
-- ------------------------------------------------------------------
joined_data AS (
    SELECT 
        pbp.*,
        sg.net_gravity,
        CASE 
            WHEN sg.net_gravity < -4 THEN 'Starbucks Death Zone'
            WHEN sg.net_gravity < 0 THEN 'Starbucks Zone'
            ELSE 'Dunkin Zone' 
        END as gravity_zone
    FROM `stuperlatives.pbp_data` pbp
    -- Join Gravity to Home Team (Environmental Factor)
    JOIN team_mapping tm ON tm.abbr = pbp.home_team
    JOIN stadium_gravity sg ON sg.team_name = tm.name
    WHERE pbp.season = 2025
      AND pbp.season_type IN ('REG', 'POST')
),

-- ------------------------------------------------------------------
-- ANALYSIS 1: PATRIOTS OFFENSE (AWAY GAMES)
-- ------------------------------------------------------------------
metrics_pats AS (
    SELECT
        'Patriots Offense (Away)' as segment,
        CASE WHEN net_gravity > 0 THEN 'Dunkin Zone' ELSE 'Starbucks Zone' END as zone,
        COUNT(DISTINCT game_id) as games,
        ROUND(AVG(CAST(complete_pass AS INT64)) / NULLIF(AVG(pass_attempt),0) * 100, 1) as comp_pct,
        ROUND(SUM(CASE WHEN play_type='pass' AND sack=0 THEN yards_gained ELSE 0 END) / NULLIF(SUM(pass_attempt),0), 2) as ypa,
        ROUND(SUM(epa) / COUNT(*), 3) as epa_per_play,
        ROUND(AVG(home_score + away_score), 1) as game_ev_ppg
    FROM joined_data
    WHERE posteam = 'NE' 
      AND home_team != 'NE' -- Away Only
      AND play_type IN ('run', 'pass')
    GROUP BY 1, 2
)
SELECT * FROM metrics_pats ORDER BY zone;

-- ------------------------------------------------------------------
-- ANALYSIS 2: SEAHAWKS DEFENSE (AWAY GAMES)
-- ------------------------------------------------------------------
SELECT 
    'Seahawks Defense (Away)' as segment,
    CASE 
        WHEN net_gravity < -4 THEN 'Starbucks Death Zone'
        WHEN net_gravity < 0 THEN 'Starbucks Zone'
        ELSE 'Dunkin Zone' 
    END as zone,
    COUNT(DISTINCT game_id) as games,
    ROUND(AVG(home_score), 1) as ppg_allowed_corrected, -- In away games, allowing home_score
    SUM(pass_touchdown) as opp_tds,
    SUM(interception) as opp_ints,
    ROUND(SUM(sack) / NULLIF(SUM(pass_attempt + sack), 0) * 100, 1) as sack_rate_pct
FROM joined_data
WHERE defteam = 'SEA'
  AND home_team != 'SEA' -- Away Only
GROUP BY 1, 2
ORDER BY zone;

-- ------------------------------------------------------------------
-- ANALYSIS 3: SAM DARNOLD (SEA QB) PARADOX (AWAY GAMES)
-- ------------------------------------------------------------------
SELECT
    'Sam Darnold (Away)' as segment,
    CASE WHEN net_gravity > 0 THEN 'Dunkin Zone' ELSE 'Starbucks Zone' END as zone,
    COUNT(DISTINCT game_id) as games,
    ROUND(SUM(pass_touchdown) / NULLIF(SUM(interception), 0.01), 2) as td_int_ratio,
    -- Passer Rating Formula Components
    ROUND(
        (
            GREATEST(0, LEAST(2.375, (SUM(complete_pass)/SUM(pass_attempt) - 0.3) * 5)) +
            GREATEST(0, LEAST(2.375, (SUM(CASE WHEN play_type='pass' AND sack=0 THEN yards_gained ELSE 0 END)/SUM(pass_attempt) - 3) * 0.25)) +
            GREATEST(0, LEAST(2.375, (SUM(pass_touchdown)/SUM(pass_attempt)) * 20)) +
            GREATEST(0, LEAST(2.375, 2.375 - (SUM(interception)/SUM(pass_attempt) * 25)))
        ) / 6 * 100
    , 1) as rating
FROM joined_data
WHERE passer_player_name = 'S.Darnold'
  AND home_team != posteam -- Away Only
GROUP BY 1, 2
ORDER BY zone;
\end{lstlisting}

\end{document}
